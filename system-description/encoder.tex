\section{Encoder (Peter)}
Each block has a table listing the input and output signals. Input signals are denoted with a left arrow ($\leftarrow$) and output signals with a right arrow ($\rightarrow$). All the blocks are going to be implemented in VHDL.
\begin{table}[H]
\centering%
\caption{Encoder signals}\label{tab:table}
\begin{tabular*}{\textwidth}{ll@{\hspace{1cm}}l}
   & \textbf{Signal} & \textbf{Description}\\\hline\hline
   $\leftarrow$ & audio & Audio data\\\hline
   $\leftarrow$ & audio\_clk & Clock for \emph{audio}\\\hline
   $\leftarrow$ & video & Video data\\\hline
   $\leftarrow$ & video\_clk & Clock for \emph{video}\\\hline
   $\rightarrow$ & i & Output 1\\\hline
   $\rightarrow$ & q & Output 2\\\hline
   $\leftarrow$ & clk & System clock\\
\end{tabular*}
\end{table}
The encoder is made up of several internal components. We’ve separated it into parts where we think it makes sense. There’s no reason to mix up unrelated logic.

\subsection{Async FIFO}
\begin{table}[!h]
\centering%
\caption{Async FIFO signals}\label{tab:table}
\begin{tabular*}{\textwidth}{ll@{\hspace{1cm}}l}
  & \textbf{Signal} & \textbf{Description}\\\hline\hline
  $\leftarrow$ & data\_in & Ingoing data\\\hline
  $\leftarrow$ & clk\_in & Clock for \textit{data\_in}\\\hline
  $\leftarrow$ & data\_out & Outgoing data\\\hline
  $\rightarrow$ & clk\_out & Clock for \textit{data\_out}\\
\end{tabular*}
\end{table}
The async FIFOs are the first block in the encoder and thus the blocks to receive the very first input signals.

There are two async FIFO blocks; one for input from audio and one for input from video. FIFO stands for “First In, First Out” and its objective is to accept input and output in the same order.

We will be using asynchronous FIFOs, meaning that it takes a clock for input and a clock for output. This enables us to easily use the asynchronous FIFOs for input synchronisation purposes. This is needed because we will be receiving audio and video data at different clocks.

\subsection{ID generator}
\begin{table}[!h]
\centering%
\caption{ID generator signals}\label{tab:table}
\begin{tabular*}{\textwidth}{ll@{\hspace{1cm}}l}
  & \textbf{Signal} & \textbf{Description}\\\hline\hline
  $\leftarrow$ & clk & Clock that increments \textit{value} everytime it hits\\\hline
  $\rightarrow$ & value & Current value\\
\end{tabular*}
\end{table}
Every frame needs an ID in the form of a number, the capacity of which has to be chosen with care. When a frame is sent with an ID, it should be received and processed before another frame is sent with the same ID.

It’s hard to definitively select a capacity for the ID, but we can calculate the allowed delay. If every frame has a size of f bits, n bits is used for ID and we transmit at t bit/s the maximum delay D is given by:

\begin{equation}
  D = \frac{2^n}{\frac{t}{f}} = \frac{f \cdot 2^n}{t}
\end{equation}

We are using a frame size of 4096 bits and aim at a transmission speed of 1.5 Mbit/s. If we use 16 bits for storing the ID we get:

\begin{equation}
  D = \frac{4096 \cdot 2^{16}}{1.5 \cdot 10^6} = 178.96 s
\end{equation}

Even for long range communication, this should be enough by far. Seeing as we have more than enough space available in the header, using 16 bits shouldn’t cause any issues.

The ID generator expects an input clock through the clock signal. At all times it keeps the current value in the value signals. The value starts at $0_{10}$ and is incremented by $1_{10}$ every time the clock strikes.

\subsection{Framer}
\begin{table}[!h]
\centering%
\caption{Framer signals}\label{tab:table}
\begin{tabular*}{\textwidth}{ll@{\hspace{1cm}}l}
  & \textbf{Signal} & \textbf{Description}\\\hline\hline
  $\leftarrow$ & clk & System clock\\\hline
  $\leftarrow$ & audio\_data & Audio data\\\hline
  $\leftarrow$ & audio\_count & Number of audio bits currently available\\\hline
  $\leftarrow$ & video\_data & Video data\\\hline
  $\leftarrow$ & video\_count & Number of video bits currently available\\\hline
  $\leftarrow$ & parity\_data & Parity data\\\hline
  $\rightarrow$ & audio\_clk & Clock for \textit{audio\_data}\\\hline
  $\rightarrow$ & video\_clk & Clock for \textit{video\_data}\\\hline
  $\rightarrow$ & parity\_clk & Clock for \textit{parity\_data}\\\hline
  $\rightarrow$ & id\_clk & Clock for the ID generator\\\hline
  $\rightarrow$ & out\_data & Output of constructed frame data\\
\end{tabular*}
\end{table}
The framer component is responsible for the actual construction of the frame and thus it is right in the middle of everything. It runs on the system clock, which it delegates around to the other components as needed.

1 bit of the frame will be filled per clock cycle. Every time a bit is completed it will be send to the splitter. Every time an information or header bit is completed it will be sent to the parity generator with a clock strike so that it can begin generating the parity bits.

As the frame construction begins, a calculation of audio and video pointers must be started. The value of the audio\_count and video\_count signals are saved. These two values are the amount of audio and video bits the frame being built will contain. Any bits given to the FIFO after this will be ignored until the construction of the next frame.

As we have 3249 information bits in each frame, each pointer must be a binary number with a length of 12. Given an ID with length 16, and two pointers, each with length 12, the "real" information bits starts at index 39.

As such the audio pointer will be the result of the addition of 39 and the saved audio\_count value. The video pointer will be the result of the addition of the audio pointer and the saved video\_count value. These additions have to be done in less than 16 clock cycles if we're not to delay the frame construction.

The first part of the frame to construct is the ID\. As described in the ID generator section this is a binary number with a length of 16. Once the 16 bits has been read and output, a clock strike will be sent through the id\_clk signal, so the ID generator can increment the ID number.

After this the audio pointer will be output followed by the video pointer. This concludes the header data and marks the start of the information data. After this the audio data is output followed by the video data.

Any leftover space in the frame must be filled by pseudo-random data.

\subsection{Parity calculator}
\begin{table}[!h]
\centering%
\caption{Parity calculator signals}\label{tab:table}
\begin{tabular*}{\textwidth}{ll@{\hspace{1cm}}l}
  & \textbf{Signal} & \textbf{Description}\\\hline\hline
  $\leftarrow$ & clk & System and input clock\\\hline
  $\leftarrow$ & in\_data & Ingoing data\\\hline
  $\rightarrow$ & out\_data & Outgoing data\\
\end{tabular*}
\end{table}
The objective of the parity generator is to calculate the parity bits of the frame. Using cyclic hamming codes represented as polynomials, this can be done in a sort of streaming way.

The parity calculator needs to contain several identical sub components for calculating groups of parity bits, as one group of parity bits for the columns can’t be completely calculated before the next group. That is however possible for the rows.

The parity calculator needs 65 sub components. 1 for the horizontal groups and 64 for the vertical groups. This is instead of having to store all the information bits of one frame.

The actual calculation will be done by finding the remainder of a polynomial division.

The output should be selected from the right subcomponent, which is possible by counting the clocks.

\subsection{Splitter}
\begin{table}[!h]
\centering%
\caption{Splitter signals}\label{tab:table}
\begin{tabular*}{\textwidth}{ll@{\hspace{1cm}}l}
  & \textbf{Signal} & \textbf{Description}\\\hline\hline
  $\leftarrow$ & in\_data & Ingoing data\\\hline
  $\rightarrow$ & out\_data & Outgoing data\\
\end{tabular*}
\end{table}
The I and Q stream is constructed from the output of the Frame block. The I stream is made from all the odd positions of the stream and the Q stream is made from all the even positions. These signals will be sent along with a clock out of the system. It is not clear yet how this is to be done.
